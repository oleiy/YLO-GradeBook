\chapter{Wprowadzenie}
\label{cha:wprowadzenie}

\textbf{YLO GradeBook} to rozbudowana aplikacja desktopowa realizująca projekt dziennika elektronicznego, stworzona w języku Java z wykorzystaniem biblioteki JavaFX. Do przechowywania danych wykorzystano relacyjną bazę danych MySQL zarządzaną przez środowisko phpMyAdmin. Główny cel aplikacji to zarządzanie ocenami uczniów, jednak została ona wzbogacona również w wiele innych funkcji, takich jak notatki osobiste, zarządzanie terminami, uwagami oraz wyświetlanie statystyk.

Projekt został stworzony z myślą o nowoczesnym i przejrzystym interfejsie graficznym użytkownika oraz odpowiednio zaprojektowanej, logicznej strukturze danych dostosowanej do poszczególnych ról użytkowników. Aplikacja posiada okno logowania oraz możliwość zmiany danych logowania użytkownika.

\vspace{1em}

\textbf{YLO GradeBook} is a developed desktop application implementing an electronic gradebook project, created in Java using the JavaFX library. A MySQL relational database managed via the phpMyAdmin environment was used to store data. The main goal of the application is to manage student grades, but it has also been enhanced with many other features such as personal notes, deadline management, comments, and displaying statistics.

The project was created with a modern and transparent graphical user interface and a properly designed, logical data structure tailored to individual user roles in mind. The application has a login window and the ability to change user login details.
