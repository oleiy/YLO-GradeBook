\chapter{Instrukcja uruchomienia plikacji YLO GradeBook}
\label{cha:instrukcja}


\section{Wymagania środowiskowe}

Do poprawnego uruchomienia aplikacji \texttt{YLO GradeBook} wymagane jest przygotowanie środowiska zgodnie z poniższymi specyfikacjami programowymi:

\begin{itemize}
    \item \textbf{Środowisko JDK:} \texttt{OpenJDK 23.0.2} (Oracle)
    \item \textbf{JavaFX SDK:} wersja 23 (dołączona lokalnie do projektu)
    \item \textbf{IDE:} IntelliJ IDEA Community Edition 2024.3.3
    \item \textbf{Sterownik JDBC:} \texttt{mysql-connector-j-9.3.0}
    \item \textbf{Baza danych:} MariaDB 10.4.32 (środowisko \texttt{phpMyAdmin 5.2.1})
    \item \textbf{Panel sterowania:} \texttt{XAMPP Control Panel 3.3.0}
\end{itemize}

\textbf{Uwaga:} Domyślnym użytkownikiem serwera bazy danych jest \texttt{root}, a hasło pozostaje puste (\texttt{""}), zgodnie z domyślną konfiguracją XAMPP.

\subsubsection*{Zainstalowane wtyczki środowiska IntelliJ IDEA}

W trakcie pracy nad aplikacją wykorzystano kilka rozszerzeń środowiska IntelliJ IDEA. Poniżej zestawiono je z podziałem na wtyczki wymagane do poprawnego działania aplikacji oraz te, które służyły jedynie do wsparcia procesu implementacji lub dokumentacji.

\textbf{Wymagane do działania aplikacji:}
\begin{itemize}
    \item \textbf{JavaFX Runtime for Plugins} — wersja \texttt{1.0.4} \\
    (Źródło: JetBrains) \\
    Wtyczka niezbędna do uruchamiania komponentów \texttt{JavaFX}w środowisku IntelliJ.
\end{itemize}

\textbf{Wykorzystane pomocniczo (opcjonalne):}
\begin{itemize}
    \item \textbf{PlantUML Integration} — wersja \texttt{7.11.2-IJ2023.2} \\
    (Autorzy: Eugene Steinberg, Vojtech Krasa) \\
    Umożliwiła stworzenie diagramu klas systemu w postaci graficznej (rozdział 3).

    \item \textbf{PlantUML Parser} — wersja \texttt{0.0.9} \\
    (Autor: shuzijun) \\
    Parser współpracujący z powyższą wtyczką, pomocny przy generowaniu diagramu UML.

    \item \textbf{GitHub Copilot} — wersja \texttt{1.5.46-243} \\
    (Źródło: GitHub) \\
    Używana jako wsparcie przy pisaniu kodu, jednak nie jest wymagana do kompilacji ani działania aplikacji.
\end{itemize}


\section{Baza danych}

\begin{itemize}
    \item Do repozytorium dołączony jest plik \texttt{gradebook\_data\_base.sql}, który zawiera strukturę i przykładowe dane bazy danych.
    \item Po uruchomieniu serwera MySQL, należy zaimportować bazę danych do środowiska \texttt{phpMyAdmin}.
    \item Domyślny użytkownik: \texttt{root} \quad Hasło: *(puste lub zgodne z konfiguracją lokalną)*
    \item Po zaimportowaniu bazy danych należy upewnić się, że jej nazwa to \texttt{gradebook\_data\_base}, zgodnie z konfiguracją aplikacji w klasie DataBaseConnection.
\end{itemize}

\section{Kroki uruchomienia projektu YLO GradeBook}

Poniżej przedsatwiono listę kroków, prowadzącą do uruchomienia aplikacji.

\vspace{6pt}
\textbf{1. Klonowanie projektu z GitHub}

\begin{enumerate}
    \item Przejdź do repozytorium projektu: \\
    \texttt{https://github.com/oleiy/YLO-GradeBook}
    \item Pobierz repozytorium lokalnie, wykonując: \\
    \texttt{git clone https://github.com/oleiy/YLO-GradeBook}
    \item Otwórz folder projektu w środowisku \texttt{IntelliJ IDEA Community Edition 2024.3.3}
\end{enumerate}

\vspace{6pt}
\textbf{2. Konfiguracja środowiska Java i JavaFX}

\begin{enumerate}
    \item Skonfiguruj środowisko zgodnie z powyższą specyfikacją.
\end{enumerate}

\vspace{6pt}
\textbf{3. Dodanie sterownika JDBC}

\begin{enumerate}
    \item Umieść folder \texttt{mysql-connector-j-9.3.0} w katalogu projektu (dołączony do repozytorium)
    \item Dodaj go jako bibliotekę do projektu w \texttt{Project Structure → Modules}
\end{enumerate}

\vspace{6pt}
\textbf{4. Import bazy danych MySQL (phpMyAdmin)}

\begin{itemize}
    \item Uruchom środowisko \texttt{XAMPP Control Panel 3.3.0} i wystartuj moduł \texttt{MySQL}
    \item Otwórz \texttt{phpMyAdmin} w przeglądarce (\texttt{http://localhost/phpmyadmin})
    \item Utwórz nową bazę danych o nazwie \texttt{gradebook\_data\_base}
    \item Zaimportuj plik \texttt{gradebook\_data\_base.sql} z repozytorium
    \item Domyślny użytkownik: \texttt{root}, hasło: (puste)
\end{itemize}

\vspace{6pt}
\textbf{5. Uruchomienie aplikacji}

\begin{enumerate}
    \item Kliknij przycisk \texttt{Run} znajdując się w klasie Main.java
    \item Aplikacja otworzy ekran logowania
\end{enumerate}

\vspace{6pt}
\textbf{6. Dane testowe (do logowania)}

\begin{itemize}
    \item \textbf{Nauczyciel:} 
    \begin{itemize}
    \item nazwa użytkownika:\texttt{jan.nowak} 
    \item hasło: \texttt{jan}
    \end{itemize}
    \item \textbf{Uczeń:} 
    \begin{itemize}
    \item nazwa użytkownika:\texttt{igor.lis} 
    \item hasło: \texttt{igor}
    \end{itemize}
\end{itemize}


\vspace{6pt}
Po wykonaniu powyższych kroków środowisko programistyczne jest gotowe do pracy projektu YLO GradeBook. W przypadku trudności zaleca się weryfikację konfiguracji JavaFX oraz parametrów połączenia z bazą danych w klasie DataBaseConnection.