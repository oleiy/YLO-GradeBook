\chapter{Podsumowanie}
\label{cha:podsumowanie}

Celem projektu było stworzenie nowoczesnej, desktopowej aplikacji \texttt{YLO GradeBook}, która w przystępny sposób umożliwia zarządzanie ocenami uczniów pozwoli na organizaje pracy szkolnej placówki. Dzięki zastosowaniu języka \texttt{Java}, biblioteki \texttt{JavaFX} oraz relacyjnej bazy danych \texttt{MySQL}, udało się zaprojektować system wyróżniający się przejrzystością interfejsu, intuicyjnością obsługi oraz rozszerzalnością struktury danych.

Zrealizowane funkcjonalności umożliwiają:
\begin{itemize}
    \item logowanie użytkowników z rozróżnieniem ról (nauczyciel/uczeń),
    \item zarządzanie ocenami, terminami, notatkami i uwagami,
    \item dynamiczne wyświetlanie danych w zależności od kontekstu,
    \item edycję danych konta oraz resetowanie hasła,
    \item wyświetlanie średnich ocen,
    \item komfortową obsługę poprzez nowoczesny interfejs oparty o komponenty graficzne FXML oraz stylizację CSS.
\end{itemize}

Proces tworzenia aplikacji przebiegał etapowo: od analizy wymagań, przez projektowanie graficzne i implementację logiki, aż po testowanie oraz przygotowanie dokumentacji. 

Efekt końcowy spełnia założone cele, a dzięki przemyślanej strukturze kodu i bazy danych, możliwa jest dalsza rozbudowa aplikacji. Przykładowe opcje rozwoju:
\begin{itemize}
    \item dodanie weryfikacji dwuetapowej podczas logowania i resetowania hasła,
    \item zintegrowanie systemu z usługami poczty elektronicznej,
    \item rozszerzenie funkcjonalności o system obecności lub komunikator wewnętrzny,
\end{itemize}

Aplikacja \texttt{YLO GradeBook} stanowi stabilną i gotową do wdrożenia podstawę elektronicznego dziennika, ukierunkowaną na prostotę, czytelność i dopasowanie do realnych potrzeb użytkowników niezależnie od ich doświadczenia w pracy z komputerem.
