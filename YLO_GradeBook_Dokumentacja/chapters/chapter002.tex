\chapter{Opis założeń projektu}
\label{cha:opisZałożeńProjektu}

%---------------------------------------------------------------------------

\section{Cel projektu}
\label{sec:celProjektu}
Celem projektu jest opracowanie aplikacji desktopowej w języku Java, pełniącej funkcję dziennika elektroniczny. System umożliwia intuicyjne zarządzanie ocenami oraz terminami, a także pozwala na prowadzenie własnych notatek i wyświetlaniem statystyk. Po zalogowaniu do systemu ukazuje się przejrzysty i nowoczesny interfejs graficzny użytkownika, zaprojektowany z myślą o estetyce i prostocie. Dla nauczyciela i ucznia przygotowano osobny widok, który przemyślano, w taki sposób, aby funkcje dostosowane do ich roli były łatwo dostępne i komfortowe w obsłudze.

Projekt YLO GradeBook odpowiada na problem braku nowoczesnych, elastycznych i prostych w obsłudze aplikacji desktopowych działających jako dziennik elektroniczny. Dostępne na rynku rozwiązania często przeładowane są zbędnymi funkcjami i nie spełniają wymagań i potrzeb dzisiejszego użytkownika, co utrudnia codzienną pracę. 

YLO GradeBook cechuje się przejrzystym interfejsem, który zachęca użytkownika do korzystania z aplikacji. W przeciwieństwie do wielu istniejących rozwiązań posiadających przestarzały interfejs, aplikacja oferuje dokładnie to, co niezbędne. Projekt realizować ma najpotrzebniejsze funkcje dziennika elektronicznego w sposób jasny i przejrzysty dla każdego użytkownika, niezależnie od jego doświadczenia z pracą przy komputerze co umożliwia łatwe korzystanie z aplikacji nawet dla osób, które nie są przekonane do elektronicznego rozwiązania omawianego problemu. Dodawanie ocen czy terminów odbywa się w intuicyjny sposób, aby realizacja zadania była szybka, komfortowa i bezproblemowa. Dodatkowo system został zaprojektowany w taki sposób, aby w przyszłości możliwe było wprowadzenie nowych funkcjonalności zgodnych z potrzebami placówek edukacyjnych.
%---------------------------------------------------------------------------

\section{Sposób realizacji projektu}
\label{sec:SposóbRealizacjiProjektu}

Realizacja projektu przebiegać będzie etapowo:
\begin{itemize}
      \item Analiza wymagań oraz zaprojektowanie najpotrzebniejszej funkcjonalności,
      \item Zaprojektowanie graficznego interfejsu na podstawie przyjętych założeń,
      \item Zaprojektowanie struktury bazy danych (MySQL), tak aby wszystkie dane przechowywane były w sposób klarowny i optymalny dla realizowanych funkcji,
      \item Stworzenie graficznego interfejsu z wykorzystaniem interfejsu JavaFX,
      \item Implementacja logiki aplikacji i połączenie z bazą danych,
      \item Testowanie funkcjonalności oraz obsługa ewentualnych błędów,
      \item Opracowanie dokumentacji projektu zawierającej opis założeń, celu, architektury i sposobu korzystania z aplikacji.
\end{itemize}
Wynikiem realizacji projektu będzie w pełni działająca i zgodna z założeniami aplikacja desktopowa YLO GradeBook umożliwiająca prowadzenie dziennika elektronicznego na potrzeby szkół. Projekt kładzie nacisk na spersonalizowaną funkcjonalność, czytelność i intuicyjną obsługę dostosowaną do potrzeb zarówno nauczycieli, jak i uczniów.

\section{Wymagania funkcjonalne}
\label{sec:WymaganiaFunkcjonalne}
W projekcie zastosowano podział na dwie grupy użytkowników, które posiadają przypisane uprawnienia i funkcjonalności.

\subsection{Logowanie}
\label{sec:Logowanie}
\begin{itemize}
      \item System umożliwia logowanie się do aplikacji za pomocą unikalnej nazwy użytkownika i hasła przechowywanego w bazie danych. Podczas logowania hasło jest ukryte, natomiast użytkownik może je chwilowo ukazać, korzystając z intuicyjnego przycisku do wyświetlenia hasła.
      \item Po zalogowaniu się system rozpoznaje, do której grupy należy użytkownik i otwiera odpowiedni interfejs graficzny dostosowany do jego roli.
      \item System od momentu zalogowania przechowuje aktualnie zalogowanego użytkownika, aż do momentu zamknięcia aplikacji.
      \item Użytkownik może zmienić swoje hasło (w przyszłości funkcja ta zostanie wyposażona w odpowiednie zabezpieczenia).
\end{itemize}

\subsection{Funkcjonalność dla grupy nauczyciel}
\label{nauczyciel}
Nauczyciel posiada dostęp do rozszerzonych funkcji administracyjnych: 

\begin{itemize}
      \item Możliwość logowania się do systemu za pomocą unikalnej nazwy użytkownika oraz hasła (na potrzeby realizacji projektu założono, iż konta użytkowników istnieją już w bazie danych)
      \item Przełączanie widoków aplikacji za pomocą przycisków w panelu nawigacyjnym po lewej stronie interfejsu,
      \item Wyświetlanie listy uczniów przypisanych do wybranej klasy,
      \item Przeglądanie ocen uczniów wybranej klasy oraz dla wybranego przedmiotu wraz z obliczeniem średnich,
      \item Przeglądanie osobistych notatek oraz możliwość ich usuwania,
      \item Przeglądanie danych osobistych przypisanych do jego konta,
      \item Możliwość zmiany hasła,
      \item Dodawanie ocen dla wybranych uczniów konkretnej klasy,
      \item Dodawania swoich notatek osobistych,
      \item Dodawanie terminów wydarzeń dla wybranej klasy,
      \item Dodawania uwag dla wybranego ucznia
      \item Możliwość wylogowania się z systemu.
\end{itemize}

\subsection{Funkcjonalność dla grupy uczeń}
\label{uczeń}
Uczeń posiada ograniczoną funkcjonalność w stosunku do nauczyciela. Posiada jednakże możliwości, które zostały zaimplementowane tylko i wyłącznie dla jego roli:

\begin{itemize}
      \item Możliwość logowania się do systemu za pomocą unikalnej nazwy użytkownika oraz hasła (na potrzeby realizacji projektu założono iż konta użytkowników istnieją już w bazie danych)
      \item Przełączanie widoków aplikacji za pomocą przycisków w panelu nawigacyjnym po lewej stronie interfejsu,
      \item Wyświetlanie powiadomień związanych z nowymi ocenami oraz terminami, które zostały dodane w przeciągu ostatnich 7 dni,
      \item Wyświetlanie ogólnej średniej ucznia,
      \item Wyświetlanie ocen, które zostały dodane w przeciągu ostatnich 7 dni,
      \item Przeglądanie wszystkich ocen z podziałem na przedmioty oraz wyliczenie średniej dla każdego z nich,
      \item Przeglądanie osobistych notatek oraz możliwość ich usuwania,
      \item Przeglądanie terminów wydarzeń, które zostały przydzielone do klasy, do której należy,
      \item Przeglądanie uwag, które zostały przypisane przez nauczyciela do jego konta,
      \item Dodawania swoich notatek osobistych,
      \item Przeglądanie danych osobistych przypisanych do jego konta,
      \item Możliwość zmiany hasła,
      \item Możliwość wylogowania się z systemu.
\end{itemize}

\subsection{Opis poszczególnych danych}
\label{opisDanych}
Dane zostały zaprojektowane, tak aby posiadały swoje cechy, które umożliwiają ich rozróżnianie, sortowanie oraz grupowanie. Dzięki temu aplikacja zachowuje sprawną organizację informacji i realizację funkcji.


\subsubsection{Oceny}
\label{oceny}
Podczas dodawania ocen przez nauczyciela pierwszym krokiem jest wybranie klasy, a następnie ucznia, któremu przypisana zostanie ocena. Następnie wybierany jest przedmiot z listy, która znajduje się w bazie danych oraz typ, gdzie do wyboru są opcje takie jak: sprawdzian, kartkówka, odpowiedź ustna, zadanie, inne. Ostatnim elementem jest wybranie oceny w systemie 1.0 - 6.0. 

Ocena składa się następujących elementów:
\begin{itemize}
      \item Klasa,
      \item Uczeń,
      \item Przedmiot,
      \item Typ,
      \item Wartość oceny.
\end{itemize}

\subsubsection{Notatki osobiste}
\label{notatki}
Użytkownik niezależnie od swojej roli może dodawać swoje notatki. W pierwszym kroku podaje tytuł notatki, a następnie jej treść. Po jej dodaniu notatka umieszczana jest w bazie danych i zostaje automatycznie przypisana do zalogowanego użytkownika bez konieczności wprowadzania jej właściciela.

Notatka zawiera:
\begin{itemize}
      \item Tytuł,
      \item Treść,
      \item Właściciela (ustalanego automatycznie).
\end{itemize}

\subsubsection{Uwagi}
\label{uwagi}
Dodając uwagę, nauczyciel wybiera klasę, a następnie ucznia, którego uwaga dotyczy. Następnie wprowadza liczbę punktów (ujemnych lub dodatnich w zależności od tego, czy uwaga jest pozytywna czy negatywna). Ostatnim elementem jest treść uwagi, która ograniczona jest do 60 znaków. 

Uwaga zawiera:
\begin{itemize}
      \item Klasę,
      \item Ucznia,
      \item Liczbę punktów (ujemnych lub dodatnich)
      \item Treśći uwagi.
\end{itemize}

\subsubsection{Terminy}
\label{terminy}
Nauczyciel, dodając nowy termin, proszony jest o wybór klasy, do której chciałby przypisać nowe wydarzenie. Następnie wybiera przedmiot z listy, która znajduje się w bazie danych, W kolejnym kroku wybiera typ (wydarzenie, sprawdzian, kartkówka, zadanie). Podaje datę wydarzenia, korzystając z pomocniczej kontrolki oraz podaje krótki opis.Uwaga zawiera:

Termin składa się z następujących elementów:
\begin{itemize}
      \item Klasa,
      \item Przedmiot,
      \item Typ wydarzenia,
      \item Data końcowa,
      \item Opis.
\end{itemize}

\newpage
\section{Wymagania niefunkcjonalne}
\label{wymaganiaNiefunkcjonalne}

\subsection{Środowisko testowe}
\label{środowiskoTestowe}
Testy projektu przeprowadzone zostały w środowisku o następującej konfiguracji:
\begin{itemize}
      \item System operacyjny: Windows 11 Pro
      \item Processor: Intel® Core™ i5-13420H 2.10Ghz
      \item Pamięć RAM: 16GB DDR4
      \item Karta graficzna: NVIDIA GeForce RTX 4050 Laptop GPU
      \item Dysk: SSD 500GB
      \item Java: openjdk-23 (Oracle OpenJDK 23.0.2)
      \item Połączenie z bazą danych: mysql-connector-j-9.3.0
      \item Baza danych: mySQL (środowisko phpMyAdmin – XAMPP Control Panel 3.3.0):
      \begin{itemize}
            \item Serwer: 127.0.0.1 via TCP/IP
            \item Typ serwera: MariaDB
            \item Połączenie z serwerem: SSL nie jest używany  
            \item Wersja serwera: 10.4.32-MariaDB - mariadb.org binary distribution
            \item Wersja protokołu: 10
            \item Użytkownik: root@localhost
            \item Kodowanie znaków serwera: UTF-8 Unicode (utf8mb4)
            \item phpMyAdmin - Informacja o wersji: 5.2.1
      \end{itemize}
\end{itemize}

\subsection{Wydajność}
\label{wydajność}
Aplikacja  została stworzona z myślą o komfortowym działaniu i płynnej obsłudze użytkownika. Przeprowadzone testy wykazały następujące cechy:
\begin{itemize}
      \item Czas uruchamiania aplikacji wynosi poniżej 3 sekund,
      \item Czas logowania nie przekracza jednej sekundy,
      \item Interfejs graficzny nie wykazuje żadnych opóźnień związanych z przełączaniem widoków lub przechodzeniem pomiędzy poszczególnymi funkcjami,
      \item Aplikacja jest stabilna i obsługuje kompletny scenariusz użytkowania nie zależnie od roli bez błędów. Wszystkie funkcje wykonywane są płynnie i komfortowo.
      \item Wyskakujące okno podczas otwiera się płynnie i nie obciąża komputera.
      \item Po zalogowaniu aplikacja zużywa około 250 MB pamięci RAM.
\end{itemize}
System został zoptymalizowany pod kątem efektywnej pracy, a jego struktura pozwala na odpowiednią organizację danych oraz funkcjonalność. Zaprojektowana architektura pozwala na łatwe dodawania nowych funkcjonalności bez naruszania obecnej struktury.

\subsection{Bezpieczeństwo}
\label{bezpieczeństwo}
System zawiera szereg rozwiązań zapewniających bezpieczeństwo danych oraz użytkownika podczas korzystania z aplikacji:
\begin{itemize}
      \item Każdy użytkownik posiada swoją unikalną nazwę użytkownika, co pozwala na jednoznaczną jego identyfikację
      \item Użytkownicy posiadają hasło, które pozwala na logowanie się do systemu
      \item Istnieje możliwość zmiany hasło w prosty sposób (przewidziana jest aktualizacja wprowadzająca dodatkowe zabezpieczenia podczas tego procesu)
      \item Pola hasła zostały ukryte; użytkownik posiada jednak możliwość odsłonienia haseł za pomocą intuicyjnego przycisku.
      \item Funkcje dostępne dla nauczyciela dostępne są po zalogowaniu się do systemu użytkownika posiadającego rolę nauczyciel.
      \item Wszelkie procesy (dodawanie ocen, notatek, terminów itp.) kończą się komunikatem informującym o sukcesie lub błędzie wykonania,
      \item Okna logowanie oraz zmiana hasła została wyposażona w odpowiednie alerty komunikujące przebieg działania. Dzięki temu np. podczas logowania użytkownik musi podać dokładną nazwę użytkownika oraz hasło, a podczas jego resetowania musi wprowadzić hasło dwukrotnie w celu jego potwierdzenia.
      \item W całej strukturze aplikacji zastosowano obsługę błędów.
\end{itemize}
Aplikacja stworzona została z myślą o dalszym rozwoju pod kątem zabezpieczeń, a więc nie zamyka się ona na nowe możliwości zabezpieczeń takie jak weryfikacja dwuetapowe lub zmiana hasła za pomocą wysłania odpowiedniej wiadomości na adres e-mail przypisany do konta.


\subsection{Użyteczność}
\label{użyteczność}
Aplikacja YLO GradeBook została zaprojektowana z myślą o prostocie i intuicyjności, tak aby mogły z niej swobodnie korzystać osoby o różnym poziomie doświadczenia pracy przy komputerze – zarówno dla nauczycieli, którzy przyzwyczajonych do tradycyjnych papierowych dzienników, jak i uczniów, którzy na co dzień korzystają z różnych aplikacji komputerowych.
\begin{itemize}
      \item Interfejs graficzny stworzony przy użyciu biblioteki JavaFX oparto na przejrzystym wyglądzie, z podziałem na zakładki umożliwiającymi przechodzenie pomiędzy widokami. 
      \item Panel nawigacyjny znajdujący się po lewej stronie jest klarowny i prosty w obsłudze. Takie rozwiązanie pozwala na lepszą optymalizację, ponieważ nie jest wymagane uruchamianie nowych okien aplikacji.
      \item Wszystkie kluczowe funkcje są dostępne w jednym oknie. W momencie dodawania danych do bazy pojawia się nowe okno, które nie obciąża systemu.
      \item Wszystkie procesy zostały zaopatrzone w odpowiednie komunikaty, dzięki czemu użytkownik informowany jest o przebiegu działania, które w danej chwili podejmuje, co sprawia, że użytkownik posiada większą kontrolę i bezpieczeństwo.
      \item System dostosowany jest do użytkownika w taki sposób, że najczęściej wykonywane działania są łatwo dostępne. Strona główna wyposażona jest w przyciski, które poprawiają automatyzację.
\end{itemize}
Wszystkie powyższe zabiegi sprawiają, że system jest czytelny dla każdego użytkownika, a nauczenie się go nie sprawia żadnych problemów, a korzystanie z niego to czysta przyjemność. Użyteczność aplikacji stanowi jeden z głównych atutów YLO GradeBook.

\subsection{Skalowalność}
\label{Skalowalność}
Aplikacja YLO GradeBook została zaprojektowana w sposób umożliwiający łatwą rozbudowę, aby możliwe było szybkie dostosowanie się do potrzeb użytkowników. W związku z tym struktura cechuje się skalowalnością pod względem rozbudowy.
\begin{itemize}
      \item Struktura bazy danych jest przygotowana pod kątem rozbudowy i łatwej organizacji danych. Możliwe jest dodanie nowych tabel i relacji bez konieczności ingerowania w istniejące. W zależności od wymagań szkół możliwe jest dodanie nowej funkcjonalności jak np. zbieranie informacji o obecnościach czy też konwersacje pomiędzy uczniami i nauczycielami. Struktura bazy danych jest na to gotowa, więc przykładowe aktualizacje nie będę trudne do zaimplementowania.
      \item Interfejs graficzny posiada taką architekturę, która pozwoli na dodawanie nowych zakładek, co pozwoli na nowe rozwiązania.
      \item W przyszłości możliwe jest integrowanie aplikacji z zewnętrznymi systemami jak np. połączenie z pocztą elektroniczną.
      \item Kod został zorganizowany w taki sposób, aby twórcy byli w stanie łatwo odnaleźć interesujący ich fragment i go zmodyfikować lub dodać nowe funkcjonalności.
      \item Na ten moment aplikacja nie posiada możliwości zmiany rozmiaru okien, jednak jej budowa pozwala na łatwe wprowadzenie tej opcji w przyszłości.
\end{itemize}
Dzięki tym cechom możliwa jest skalowalność aplikacji i jej dalszy rozwój pod kątem spełniania oczekiwań użytkowników. Zorganizowana architektura jest elastyczna, co sprawia, że kod nie będzie potrzebował gruntownej przebudowy.
\newpage
\subsection{Integralność danych}
\label{integralnośćDanych}
W projekcie YLO GradeBook zachowanie integralności danych stanowi jeden z kluczowych aspektów działania systemu. Wdrożono takie rozwiązania, aby zapewnić spójność danych i zminimalizować prawdopodobieństwo wystąpienia błędów.
\begin{itemize}
      \item Wszystkie dane, które wprowadzane do bazy danych przechodzą podstawową walidację po stronie aplikacji (np. wymagane pola, limity znaków, typy danych),
      \item Aplikacja posiada zautomatyzowane dodawanie danych, w taki sposób, aby użytkownik nie musiał ręcznie wpisywać niektórych informacji. Wystarczy, że wybierze on jedną z opcji, co ogranicza możliwość wprowadzania błędnych danych i konieczności walidacji,
      \item Wartości danych są ograniczone logicznie (np. oceny od 1.0 do 6.0, punkty w przypadku dodawania uwag to liczby całkowite),
      \item Baza danych wykorzystuje klucze główne i obce, co pozwalana na powiązania relacyjne między encjami,
      \item Operacje zapisu i aktualizacji w bazie danych wykorzystują odpowiednio przemyślanie zapytania w języku SQL z parametrami przekazywanymi w aplikacji. Wszystkie metody realizujące zadania tego typu zaopatrzone są w odpowiednią obsługę błędów,
      \item Dla kluczowych operacji zastosowano system wyświetlających się komunikatów o przebiegu operacji,
      \item Dzięki odpowiednio przemyślanej strukturze bazy danych możliwe jest odpowiednie pobieranie danych oraz ich wyświetlanie w zależności od spełnionych wymagań

\end{itemize}
Dzięki tym mechanizmom YLO GradeBook zapewnia wysoki poziom spójności danych, co zapewnia dokładność danych – ma to kluczowe znaczenie w środowisku szkolnym.




